\documentclass{article}
\usepackage{multicol}
\usepackage{graphicx}
\linespread{1.35}
\usepackage{amsmath}
\usepackage{color}
\usepackage{xcolor}
\usepackage{tikz}
\usetikzlibrary{arrows,automata}


\begin{document}

\begin{flushright}
 \texttt{Pushdown Automata} \hspace*{0.10cm}\textbf{$|$} \textbf{405}\hspace*{0.5cm}
\end{flushright}

\vspace*{0.5cm}
\begin{center}
   $(q_1 z_2 q_2) \rightarrow b(q_1 z_2 q_1) (q_1 z_2 q_2)$\\
   $(q_1 z_2 q_2) \rightarrow b(q_1 z_2 q_2) (q_2 z_2 q_2)$.\\
\end{center}

\fcolorbox{red}{blue}{\textbf{\textcolor[rgb]{1.00,1.00,1.00}{Example 7.28}}}\hspace*{0.1cm} \texttt{\small{Convert the following PDA into an equivalent CFG.}}\\


\hspace*{2cm} $\delta(q_0, a, z_0) \rightarrow (q_1, z_1z_0)$ \\
\hspace*{2cm} $\delta(q_0, b, z_0) \rightarrow (q_1, z_2z_0)$\\
\hspace*{2cm} $\delta(q_1, a, z_1) \rightarrow (q_1, z_1z_1)$\\
\hspace*{2cm} $\delta(q_1, b, z_1) \rightarrow  (q_1, \lambda)$\\
\hspace*{2cm} $\delta(q_1, b, z_2) \rightarrow (q_1, z_2z_2)$\\
\hspace*{2cm} $\delta(q_1, a, z_2) \rightarrow (q_1, \lambda)$\\
\hspace*{2cm} $\delta(q_1, \lambda, z_0) \rightarrow (q_1, \lambda) // accepted by the empty stack$ \\

\vspace*{0.5cm}

\textbf{Solution:} The PDA contains two states: $q_0$ and $q_1$. The following productions are added to the CFG
[according to rule (1)].\\
\begin{center}
  $S \rightarrow [q_0 z_0 q_0]/[q_0 z_0 q_1]$\\
\end{center}

\hspace*{0.5cm} Transitional functions (iv), (vi), and (vii) are in the form $\delta(q, a, Y) \rightarrow (r, \in)$. Thus, the following three
productions are added to the CFG [according to rule (2)]\\

\begin{center}
\hspace*{0.5cm} $(q_1 z_1 q_1) \rightarrow b$ // From production (iii) \\
\hspace*{0.5cm} $(q_1 z_2 q_2) \rightarrow a$ // From production (iv)\\
\hspace*{0.5cm} $(q_1 z_0 q_1) \rightarrow \varepsilon$ // From production (vii) \\
\end{center}

For the remaining transitional functions, the productions are as follows:\\


\begin{center}
$\delta(q_0, a, z_0) \rightarrow (q_1, z_1z_0)$ \\
 \hspace*{1cm} $(q_0 z_0 q_0) \rightarrow a(q_0 z_1 q_0) (q_0 z_0 q_0)$ \\
 \hspace*{1cm} $(q_0 z_0 q_0) \rightarrow a(q_0 z_1 q_1) (q_1 z_0 q_0)$ \\
 \hspace*{1cm} $(q_0 z_0 q_1) \rightarrow a(q_0 z_1 q_0) (q_0 z_0 q_1)$ \\
 \hspace*{1cm} $(q_0 z_0 q_1) \rightarrow a(q_0 z_1 q_1) (q_1 z_0 q_1)$ \\
$\delta(q0, b, z0) \rightarrow (q1, z2z0)$ \\
 \hspace*{1cm} $(q_0 z_0 q_0) \rightarrow b(q_0 z_2 q_0) (q_0 z_0 q_0)$ \\
 \hspace*{1cm} $(q_0 z_0 q_0) \rightarrow b(q_0 z_2 q_1) (q_1 z_0 q_0)$ \\
 \hspace*{1cm} $(q_0 z_0 q_1) \rightarrow b(q_0 z_2 q_0) (q_0 z_0 q_1)$ \\
 \hspace*{1cm} $(q_0 z_0 q_1) \rightarrow b(q_0 z_2 q_1) (q_1 z_0 q_1)$ \\
$\delta(q_1, a, z_1) \rightarrow (q_1, z_1z_1)$ \\
 \hspace*{1cm} $(q_1 z_1 q_0) \rightarrow a(q_1 z_1 q_0) (q_0 z_1 q_0)$ \\
 \hspace*{1cm} $(q_1 z_1 q_0) \rightarrow a(q_1 z_1 q_1) (q_1 z_1 q_0)$ \\
 \hspace*{1cm} $(q_1 z_1 q_1) \rightarrow a(q_1 z_1 q_0) (q_0 z_1 q_1)$ \\
 \hspace*{1cm} $(q_1 z_1 q_1) \rightarrow a(q_1 z_1 q_1) (q_1 z_1 q_1)$ \\
\end{center}

\newpage
\begin{flushleft}
    \textbf{406}\hspace*{0.1cm} \textbf{$|$} \hspace*{0.1cm} \texttt{Introduction to Automata Theory, Formal Languages and Computation}
  \end{flushleft}

\vspace*{0.5cm}
\begin{center}
$\delta(q_1, b, z_2) \rightarrow (q_1, z_2z_2)$ \\
 \hspace*{2cm} $(q_1 z_2 q_0) \rightarrow b(q_1 z_2 q_0) (q_0 z_2 q_0)$ \\
 \hspace*{2cm} $(q_1  z_2 q_0) \rightarrow b(q_1 z_2 q_1) (q_1 z_2 q_0)$ \\
 \hspace*{2cm} $(q_1 z_2 q_1) \rightarrow b(q_1 z_2 q_0) (q_0 z_2 q_1)$ \\
 \hspace*{2cm} $(q_1 z_2 q_1) \rightarrow b(q_1 z_2 q_1) (q_1 z_2 q_1)$ \\
\end{center}

\vspace*{0.3cm}

\large{
\textbf{7.6 Graphical Notation for PDA}\\
}

\small{The mathematical notation for a PDA is $(Q, \Sigma, \Gamma, \delta, q_0, z_0, F)$. In a PDA, the transitional function $\delta$
consists of three touples: fi rst is a present state, second is the present input, and the third is the stack top
symbol, which generates one next state and the stack symbol(s) if a symbol is pushed into the stack or
$\in$, if the top most symbol is popped from the stack.
In the graphical notation of the PDA, there are states. Among them a circle with an arrow indicates a
beginning state and a state}


 with double circle indicates a final state.The state transitions are denoted byarrows. The labels of the state transitions consists of input symbol, previous stack top symbol (at $t_i-1$) and the current stack top symbol (at $t_i$) which is added after the transitions or null symbol (if a symbol is popped).\\

\fcolorbox{red}{blue}{\textbf{\textcolor[rgb]{1.00,1.00,1.00}{Example 7.29}}}\hspace*{0.1cm} \texttt{\small{Construct a PDA with a graphical notation to accept $L = (a, b)*$ with equal number of 'a' and 'b', i.e., $n_a(L) = n_b(L)$ by the fi nal state.}}\\

\textbf{Solution: }At the beginning of transition, the PDA is in state $q_0$ with stack $z_0$. The string may start with
'a' or 'b'. If the string starts with 'a', one $z_1$ is pushed into the stack. If the string starts with 'b', one $z_2$
is pushed into the stack. If 'b' is traversed after 'a', and the stack top is $z_1$, that stack top is popped. If 'a'
is traversed after 'b', and the stack top is $z_2$, that stack top is popped. If 'a' is traversed after 'a', and the
stack top is $z_1$, one $'z_1'$ is pushed into the stack. If 'b' is traversed after 'b', and the stack top is $z_2$, one
$'z_2'$ is pushed into the stack.\\


\hspace*{0.5cm} The PDA in graphical notation is as follows:\\

\hspace*{4cm} $a, z_0/z_1z_0$ \\
\hspace*{4cm} $b, z_0/z_2z_0$ \\
\hspace*{4cm} $a, z_1/ z_1z_1$ \\
\hspace*{4cm} $b, z_2/z_2z_2$ \\
\hspace*{4cm} $b, z_1/\lambda$ \\
\hspace*{4cm} $a, z_2/\lambda$ \\


\begin{center}
\section{picture}
\includegraphics[width=5cm,height=1.5cm]{405.png}
\end{center}

\fcolorbox{red}{blue}{\textbf{\textcolor[rgb]{1.00,1.00,1.00}{Example 7.30}}}\hspace*{0.1cm} \texttt{\small{Construct a PDA with a graphical notation to accept the language L = {WCWR, where
W $\in (a, b)+$ and WR is the reverse of W} by the fi nal state.}} \\

\newpage
\begin{flushright}
 \texttt{Pushdown Automata} \hspace*{0.10cm}\textbf{$|$} \textbf{407}\hspace*{0.5cm}
\end{flushright}

\vspace*{0.5cm}
$z_1$ or $z_2$, another $z_1$ or $z_2$ is pushed into the stack, respectively. In state $q_1$, if it gets the input ‘b’ with the
stack top $z_2$ or $z_1$, another $z_2$ or $z_1$ is pushed into the stack, respectively.
\hspace*{0.5cm} C is the symbol which differentiates W with WR. Before C, W is traversed. So, at the time of traversing
C, the stack top symbol may be $z_1$ or $z_2$. In state $q_1$, if the PDA gets C as input and the stack top $z_1$ or
$z_2$, no operation is done on the stack, but only the state is changed from $q_1$ to $q_2$. After C, the string WR is
traversed. If the machine gets ‘a’ as input, the stack top must be $z_1$. And that $z_1$ is popped from the stack
top. If the machine gets 'b' as input, the stack top must be $z_2$. And that $z_2$ is popped from the stack top.
The state is not changed. By this process, the whole string WCWR is traversed. In state $q_2$, if the machine
gets no input but stack top $z_0$, the machine goes to its fi nal state $q_f$.\\

The graphical notation for the PDA is as follows:\\

\vspace*{0.3cm}
\begin{center}
\section{picture}
\includegraphics[width=8cm,height=2.5cm]{407.png}
\end{center}

\large{
\textbf{7.7 Two-stack PDA}
}

\small{
Finite automata recognize regular languages such as $\{an | n \geq 0\}$. Adding one stack to a fi nite automata, it
becomes PDA which can recognize context-free language $\{a^nb^n, n \geq 0\}$. In the case of context-sensitive
language such as $\{a^nb^n c^n, n \geq 0\}$, the PDA is helpless because of only one auxiliary storage. Now the
question arises-if more than one stack is added in the form of auxiliary storage with PDA, does its
power increase or not.}\\

\hspace*{0.5cm} From this question, the concept of a two-stack PDA has come. Not only two stacks, but more than
two stacks can be added to a PDA.\\
\hspace*{0.5cm} A PDA can be deterministic or non-deterministic, but two-stack PDA is deterministic and it accepts
all context-free languages, which may be deterministic or non-deterministic, with context-sensitive language
such as $\{a^nb^n c^n, n \geq 0\}$. In the Turing machine chapter, we shall learn that two-stack PDA is
equivalent to the Turing machine. There, we shall also learn a theorem called the Minsky's theorem.

\vspace*{0.3cm}
\textbf{Defi nition:} A two-stack PDA consists of a 9-tuple\\

\begin{center}
  $M = (Q, \Sigma, \Gamma, \Gamma', \delta, q_0, z_1, z_2, F)$ \\
\end{center}


where\\
\hspace*{1cm} Q denotes a fi nite set of states.\\
\hspace*{1cm} $\Sigma$ denotes a fi nite set of input symbols.\\
\hspace*{1cm} $\Gamma$ denotes a fi nite set of fi rst stack symbols.\\
\hspace*{1cm} $\Gamma '$ denotes a fi nite set of second stack symbols\\
\hspace*{1cm} $\delta$ denotes the transitional functions.\\
\hspace*{1cm} $q_0$ is the initial state of PDA $[q_0 \in Q]$.\\
\hspace*{1cm} $z_1$ is the stack bottom symbol of stack 1.\\
\hspace*{1cm} $z_2$ is the stack bottom symbol of stack 2.\\
\hspace*{1cm} F is the fi nal state of PDA.\\

\newpage
\begin{flushleft}
    \textbf{408}\hspace*{0.1cm} \textbf{$|$} \hspace*{0.1cm} \texttt{Introduction to Automata Theory, Formal Languages and Computation}
  \end{flushleft}

\vspace*{0.5cm}
In PDA, the transitional function $\delta$ is in the form\\

\begin{center}
  $Q \times (\Sigma \cup \{\lambda\}) \times \Gamma \times \Gamma ' \rightarrow (Q, \Gamma, \Gamma ')$ \\
\end{center}

\fcolorbox{red}{blue}{\textbf{\textcolor[rgb]{1.00,1.00,1.00}{Example 7.31}}}\hspace*{0.1cm} \texttt{\small{Construct a two-stack PDA for the language $L = \{anbncn, n \geq 0\}$.}}\\

\textbf{Solution:} While scanning ‘a’, push X into stack 1. While scanning ‘b’, push ‘Y’ into stack 2. While
scanning ‘c’ with stack top X in 1 and stack top Y in 2, pop X and Y from stack 1 and 2, respectively.\\
The transitional functions are as follows:\\

\hspace*{2.5cm} $\delta(q_0, \lambda, z_1, z_2) \rightarrow (q_f , z_1, z_2)$ \\
\hspace*{2.5cm} $\delta(q_0, a, z_1, z_2) \rightarrow (q_0, Xz_1, z_2)$ \\
\hspace*{2.5cm} $\delta(q_0, a, X, z_2) \rightarrow (q_0, XX, z_2)$ \\
\hspace*{2.5cm} $\delta(q_0, b, X, z_2) \rightarrow (q_0, X, Yz_2)$ \\
\hspace*{2.5cm} $\delta(q_0, b, X, Y) \rightarrow (q_0, X, YY)$ \\
\hspace*{2.5cm} $\delta(q_0, c, X, Y) \rightarrow (q_1, \lambda, \lambda)$ \\
\hspace*{2.5cm} $\delta(q_1, c, X, Y) \rightarrow (q_1, \lambda, \lambda)$ \\
\hspace*{2.5cm} $\delta(q_1, \lambda, z_1, z_2) \rightarrow (q_f ,, z_1, z_2)$ // accepted by the fi nal state.\\

\vspace*{0.1cm}
\textbf{Theorem 7.2:} Intersection of RE and CFL is CFL.\\

\vspace*{0.1cm}
\textbf{Proof:} Let L is a CFL accepted by a PDA $M_1 = \{Q_1, \Sigma, \Gamma_1, \delta_1, q_01, z_0, F_1\}$ and R be a regular expression
accepted by a FA $M_2 = \{Q_2, \Sigma, \delta_2, q_02, F_2\}$. A new PDA $M_3 = \{Q, \Sigma, \Gamma, \delta, q_0, z_0, F\}$ is designed
which performs computation of $M_1$ and $M_2$ in parallel and accepts a string accepted by both $M_1$ and $M_2$.
$M_3$ is designed as follows.\\


\hspace*{2.2cm} $Q = Q_1 \times Q_2$ [Cartesian product]\\
\hspace*{2.2cm} $\Sigma = \Sigma$ \\
\hspace*{2.2cm} $\Gamma = \Gamma_1$ \\
\hspace*{2.2cm} $F = F_1 \times F_2$ \\

\vspace*{0.2cm}
$\delta: [(S_1, S_2), i/p, Z) \rightarrow (q_1, q_2, Z')$ if $\delta_1(S_1,i/p, Z) \rightarrow (q_1, Z')$ and $\delta_2(S_2,i/p) \rightarrow q_2$
\hspace*{0.5cm} $[Z$ and $Z' \in \Gamma]$ \\

 \begin{center}
   $q_0 = q_01 \cup q_02$ \\
 \end{center}

\vspace*{0.2cm}
\hspace*{0.5cm} The transitional function of $M_3$ keeps track of transaction from $S_1$ to $q_1$ in PDA $M_1$ and $S_2$ to $q_2$ in FA
$M_2$ for same input alphabet. Thus a string W accepted by $M_3$ if and only if, it is accepted by both $M_1$ and
$M_2$. Therefore $W \in L(M_1) \cap L(M_2)$. As W is accepted by a PDA, W is a CFL.\\



\begin{center}
\section{picture}
\includegraphics[width=11cm,height=0.8cm]{408.png}
\end{center}

\begin{enumerate}
  \item Pushdown automata (in short PDA) is the machine format of context-free language.\\
  \item The mechanical diagram of a pushdown automata contains the input tape, reading head, fi nite
control, and a stack.\\
  \item A pushdown automata consists of a 7-tuple $M = (Q, \Sigma, \Gamma, \delta, q_0, z_0, F)$ where $Q, \Sigma, q_0$, and F have
their original meaning, $\Gamma$ is fi nite set of stack symbols, and $z_0$ is the stack bottom symbol.\\
  \item In a PDA, the transitional function $\delta$ is in the form $Q \times (\Sigma \cup {\lambda}) \times \Gamma \rightarrow (Q, \Gamma).$ \\
\end{enumerate}

\end{document} 